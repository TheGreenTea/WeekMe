\section{User Documentation}
\textit{(This section is also part of the readme.md on the weekme-github repository)} \\
\\Using \textbf{WeekMe} you can plan your next week. Not more, not less. No bloated UI nor functions you will probably never use anyway. 
How does this \textbf{awesome application} work? Hopefully it's so intuitive that any explanation is unnecessary, but just in case here is a short introduction for you.

\subsection{Account}
Like in most web apps you can sign up for an account, log in and out as well as change your password and email. And of course there is also a solution in case you forget your password. 
\begin{figure}[H] 
	\centering 
	\includegraphics[height=10cm]{figures/user_docu_accounts_page.PNG}   
	\caption[WeekMe account page]{WeekMe account page on an iPhone XS}       
	\label{fig: Setting an environment variable in Heroku}     
\end{figure}  

\subsection{Working with tasks}
Since the app is all about planing your next weeks task, let's show you how you can work with these tasks on WeekMe. 
\subsubsection{Creating tasks}
To create a new task you select the plus icon where you want to add the task. This will bring up a popup window where you can enter a short description (up to 80 characters). 
If you fancy selecting a colour for your task to help you categorise, prioritise or just because you want more colour in your life - no false modesty, go on. Do it. 
In case you changed your mind or just clicked the wrong add button you can also change the day the task will be added to. 

\begin{figure}
    \centering
    \subfloat[Popup to enter description in]{{\includegraphics[height=10cm]{figures/user_docu_createtask.PNG} }}
    \qquad
    \subfloat[Change day later on]{{\includegraphics[height=10cm]{figures/user_docu_createtask_day.PNG} }}
    \caption{Creation process on mobile}
    \label{fig:example}
\end{figure}
 
\subsubsection{Editing tasks}
In case you misspelled something, you want to rephrase something or you just have too much colour on your screen and want to make changes to a task you previously created, just select the task by clicking/tapping on it and select the handy edit button in the upper left corner. 
\subsubsection{Moving tasks}
If you want to bring tasks within one day into order, you can move them around! Just select a task and select the task you want to swap positions with afterwards. 
This also works if you want to move a task to a different day. 
\subsubsection{Checking off and deleting tasks}
You did it. You successfully planned something, stuck to the plan and finished a task. What do you do? You check it off by selecting it and clicking/tapping the checkmark icon in the top right corner. It's really that easy! \\
Don't like a task you created anymore? Just check it off as done - no one will know!
\subsubsection{The stack}
At the very bottom of your week there is an eighth entry labeled "stack" - What's that all about you might ask - well, this is where you can put tasks that you know you want to do this week but just don't exactly know on what day yet.  You can interact with it just as with every other day as explained above.  
In case you did not manage to check off everything on any given day the task will automatically be moved to the stack for you.