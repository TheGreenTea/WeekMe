\documentclass[11pt]{article}

\usepackage[utf8]{inputenc}
\usepackage{mathptmx}
\usepackage[T1]{fontenc} 

\usepackage{lipsum}
\usepackage[margin=1in,left=1.5in,includefoot]{geometry}   

\usepackage[hidelinks]{hyperref} % Allows for clickable references // hidelinks --> stops links from being formatted differently

\usepackage{color} 
\usepackage{xcolor}

% Graphics preamble 
\usepackage{graphicx} %Allows to import images 
\usepackage{float} %Allows for control of float positions 
\graphicspath {{figures/}}  
\usepackage{wrapfig}
\usepackage{subcaption}


% Text preamble
\usepackage[normalem]{ulem} %underlining 
\newcommand{\myparagraph}[1]{\paragraph{#1}\mbox{}\\} 

%Table Preamble
\usepackage{pbox}


% Source Code preamble
\usepackage{listings} %code highlighter
\usepackage{color} %use color
\definecolor{mygreen}{rgb}{0,0.6,0}
\definecolor{mygray}{rgb}{0.5,0.5,0.5}
\definecolor{mymauve}{rgb}{0.58,0,0.82}
 
%Customize a bit the look
\lstset{ %
backgroundcolor=\color{white}, % choose the background color; you must add \usepackage{color} or \usepackage{xcolor}
basicstyle=\tiny, % the size of the fonts that are used for the code  
breakatwhitespace=false, % sets if automatic breaks should only happen at whitespace
breaklines=true, % sets automatic line breaking 
captionpos=b, % sets the caption-position to bottom
commentstyle=\color{mygreen}, % comment style
deletekeywords={...}, % if you want to delete keywords from the given language
escapeinside={\%*}{*)}, % if you want to add LaTeX within your code
extendedchars=true, % lets you use non-ASCII characters; for 8-bits encodings only, does not work with UTF-8
frame=single, % adds a frame around the code
keepspaces=true, % keeps spaces in text, useful for keeping indentation of code (possibly needs columns=flexible)
keywordstyle=\color{blue}, % keyword style
% language=Octave, % the language of the code
morekeywords={*,...}, % if you want to add more keywords to the set
numbers=left, % where to put the line-numbers; possible values are (none, left, right)
numbersep=5pt, % how far the line-numbers are from the code
numberstyle=\tiny\color{mygray}, % the style that is used for the line-numbers
rulecolor=\color{black}, % if not set, the frame-color may be changed on line-breaks within not-black text (e.g. comments (green here))
showspaces=false, % show spaces everywhere adding particular underscores; it overrides 'showstringspaces'
showstringspaces=false, % underline spaces within strings only
showtabs=false, % show tabs within strings adding particular underscores
stepnumber=1, % the step between two line-numbers. If it's 1, each line will be numbered
stringstyle=\color{mymauve}, % string literal style
tabsize=2, % sets default tabsize to 2 spaces
title=\lstname % show the filename of files included with \lstinputlisting; also try caption instead of title
}
%END of listing package%
 
\definecolor{darkgray}{rgb}{.4,.4,.4}
\definecolor{purple}{rgb}{0.65, 0.12, 0.82}
 
%define Javascript language
\lstdefinelanguage{JavaScript}{
keywords={typeof, new, true, false, catch, function, return, null, catch, switch, var, if, in, while, do, else, case, break},
keywordstyle=\color{blue}\bfseries,
ndkeywords={class, export, boolean, throw, implements, import, this},
ndkeywordstyle=\color{darkgray}\bfseries,
identifierstyle=\color{black},
sensitive=false,
comment=[l]{//},
morecomment=[s]{/*}{*/},
commentstyle=\color{purple}\ttfamily,
stringstyle=\color{red}\ttfamily,
morestring=[b]',
morestring=[b]"
}
 
\lstset{
language=JavaScript,
extendedchars=true,
basicstyle=\footnotesize\ttfamily, 
showstringspaces=false,
showspaces=false,
numbers=left,
numberstyle=\footnotesize,
numbersep=9pt,
tabsize=2,
breaklines=true,
showtabs=false,
captionpos=b
}



% Table preamble
\usepackage{array}


%Math preamble
\usepackage{mhchem} %Allows us to write chemistry equations! 
\usepackage{xfrac} % Allows for slanted fractions 


% Bibliography preamble
\usepackage[numbers,sort&compress]{natbib} %Sorts references by numner, if 1,2,3 --> 1-3 is shown

% Bullet preamble 
\renewcommand{\labelitemi}{$\bullet$}  
\renewcommand{\labelitemii}{$\diamond$}  
\renewcommand{\labelitemiii}{$\circ$}      

% Header and Footer Stuff
\usepackage{fancyhdr}
\pagestyle{fancy}
\fancyhead{}
\fancyfoot{}
\fancyfoot[R]{\thepage}     
\renewcommand{\headrulewidth}{1pt}  
\renewcommand{\footrulewidth}{1pt}     
  
 

\setcounter{section}{0}  
\setcounter{tocdepth}{2}
\setcounter{secnumdepth}{4} 
 
\begin{document}

\begin{titlepage} 
	\begin{center}
	\line(1,0){300} \\
	[4mm]
	\huge{\bfseries WT1 Web Applications \\ Projekt: WeekMe} \\  
	[2mm]
	\line(1,0){200}\\	
	[1.5cm]
	\textsc{\LARGE HTW BERLIN}\\
	[0.75cm]
	\textsc{\LARGE Internationale Medieninformatik} \\  
	[9cm]
	\end{center}	 
	\begin{flushright}
		\textsc{\large Jakub Mueller / David Schmoecker / Jonas Heinig \\
		\#544832 / 544655 / 550169 \\
		WS18/19\\}
	\end{flushright} 
	 
	
	
\end{titlepage}  
 
% Front matter stuff
\pagenumbering{roman} 

% Table of Contents Stuff
\tableofcontents
  
\cleardoublepage %Make section start on next page 




% Main Body Stuff 
\pagenumbering{arabic} 
\setcounter{page}{1} %Reset page counter to 1 (to make page with first section start on page 1 instead of ...) 

\section{Technical Documentation}

\subsection{Architecture} 
Folgendes Bild gibt einen groben Überblick über die in der WeekMe Webapplikation verwendeten Technologien und wo diese im Gesamtsystem zum Einsatz kommen. 
 
The image below gives an overview about the technologies that we have used in the development of the WeekMe application and where these are used within our overall system.  

	\begin{figure}[H] 
		\centering 
		\includegraphics[height=6.2cm]{figures/architecture}    
		\caption{System Architecture}     
	\end{figure}  

\subsection{Backend} 

\subsubsection{Node.js}

\myparagraph{Modules}  



\textbf{Express.js}  is one of the most popular modules available on npm. It is a server side web framework that extends the functionality of Node.js with many handy tools making development faster and easier. An express application consists of at least one file usually called app.js. This file registers any middleware and controllers the application is using. Furthermore the web-server is started here. A basic express web-server can be started in the following way:  

\begin{lstlisting}
const express = require("express");
var app = express();
app.listen(3000, () => {
  console.log("Server is up on port: ", 3000); 
});  
\end{lstlisting}    



\textbf{Routing} 
Routing is used to determine how our application responds to client requests at a particular endpoint.  
Typical client requests HTTP methods are GET, POST, PUT and DELETE. The initial setup of our routing is done using an instance of our express app. We can add routes and their handlers to our app in the following form:

\begin{lstlisting}
	app.METHOD(PATH, HANDLER); 
\end{lstlisting}

Where app is an instance of express, METHOD is an HTTP request method, PATH is a path on the server and HANDLER is the function executed when the route is matched. The simple examples below demonstrate a basic routing for each HTTP method type. In all of these cases our handler simply returns a string value to the client.   

\begin{lstlisting}
app.get('/', function (req, res) {
  res.send('Hello World!')
})

app.post('/', function (req, res) {
  res.send('Got a POST request')
})

app.put('/user', function (req, res) {
  res.send('Got a PUT request at /user')
})

app.delete('/user', function (req, res) {
  res.send('Got a DELETE request at /user')
})

\end{lstlisting}


\textbf{Serving static files}
In order to be able to serve static files our WeekMe application needs to be configured properly. The most common scenario in which we need to serve public files is when we want to include css, html or other resource files. We use an integrated express middleware function called "express.static" (as explained in the next section) to let our app know where to find our public directory containing the static files. Having done this clients are able to load files from the public directory.     

\begin{lstlisting}
	app.use(express.static('public'));  
\end{lstlisting}



\textbf{Express Middleware} lets us run certain code before or after certain events. Express middleware adds on to the express functionality. Middleware functions have access to the request and response object as well as the next middleware function in the execution cycle.  Common examples of things express middleware can do are:  

\begin{itemize}
	\item Execute Code
	\item Change to request or response objects
	\item Determine whether or not somebody is logged in and may access a page
	\item respond to a request 
\end{itemize}

The following middleware function is used inside the application to determine whether the user that is accessing the route is logged in.  

\begin{lstlisting}
var authenticate = (req, res, next) => {
  var token = req.header("x-auth");

  User.findByToken(token).then((user) => {
    if(!user){
      return Promise.reject();
    }
    req.user = user;
    req.token = token;
    next(); 
  }).catch((e) => {
    res.status(401).send();
  });
};
\end{lstlisting}  

After having defined our middleware function, we need to append it to one of our routes so that it will be executed when this route is being accessed.   

\begin{lstlisting}
//Tell route to use middleware authenticate
app.get("/users/me", authenticate, (req, res) => {
  //because we are using the middleware, we now have all the data we appended inside that middleware available here inside the req object
  res.send(req.user);
}); 
\end{lstlisting} 

\textbf{Bcrypt} is a module that can be used to hash passwords. Hashing passwords is considered a mandatory operation before storing them to a database in any secure and professional environment, as it prevents attackers that might get access to the dataset from reading and using any passwords that are stored. Hashing is a one way street and always returns the exact same hash for the same string. When hashing a password a salt is used to increase security even further. Salting password hashes is also considered mandatory. Otherwise we will always receive the same exact hash when using the same exact password. This would enable people to use a lookup table that maps thousands of common password phrases or english words to their hashes. Ultimately they would be able to get the password from the hash. Thats why an unknown salt is needed. Often a randomly generated value is used for the salting process. This ensures that will never receive the same value twice and prevents users from being able to trick us and change data by generating their own hashes. 

\begin{lstlisting}
// Hashing a password
bcrypt.hash('my password', 'my salt', (err, hash) => {
  // Store hash password in DB
});

//Checking a password
bcrypt.compare('my password', hash, (err, res) => {
  // res == true or res == false
});
\end{lstlisting}


\textbf{JsonWebToken} is yet another npm module that helps us to setup a secure environment for our application. Its main use case is generating unique tokens that can be used for user authentication. Once generated and assigned to a specific user, the tokens can be used to access private routes that could otherwise not be accessed. This way not anyone can make a HTTP request to our server and we make sure that only verified and authenticated requests are being made. In fact, most routes inside WeekMe will be private and only the login and signup routes will be public. In the code snipped below we use jsonwebtoken to create a token. When the token is decoded using jsonwebtokens verify method, we need to pass in the exact same salt that we used when creating the token, otherwise we will not be able to access the id and iat (=issuedAt) properties that are stored inside the data object.  

\begin{lstlisting}
const jwt = require("jsonwebtoken");

var data = {
  id: 10
};

var token = jwt.sign(data, "mysaltysalt");
console.log(token);
var decoded = jwt.verify (token, "mysaltysalt");
console.log("Decoded", decoded);  //Decoded --> {id: 10, iat: 1526910769}
\end{lstlisting}


We can visit \url{https://jwt.io } to get information about the hash we have created using the jsonwebtoken module. Each component of our token stores a different information. The red part is the header. The purple part is the payload storing the actual information (id, iat) and the blue part is the hash that lets us verify that the payload has never been changed. 

	\begin{figure}[H] 
		\centering 
		\includegraphics[height=12cm]{figures/jwt}    
		\caption[Jsonwebtoken info at: https://jwt.io]{Jsonwebtoken info at: https://jwt.io}     
		\label{fig: Jsonwebtoken info at: https://jwt.io}      
	\end{figure}    

 


When a user signs up, a unique token will be created and stored into the database. From now on this token is used for any authentication when accessing private routes instead of the email and password properties that have been provided by the user in the signup process. 

\begin{lstlisting}
UserSchema.methods.generateAuthToken = function () {
  var user = this; 
  var access = "auth";
  var token = jwt.sign({_id: user._id.toHexString(), access}, process.env.JWT_SECRET);

  user.tokens = user.tokens.concat([{access, token}]);
  return user.save().then(() => {
    return token;
  });
};

\end{lstlisting}




\subsubsection{Heroku}
Heroku is a handy and powerful online platform that enables us to easily deploy and test our applications in the cloud. Once we have setup our account on Heroku.com a series of terminal commands is required in order to get our application up and running on Heroku. The first step is to login to our Heroku account inside of the terminal. 

\begin{verbatim}
	$ heroku login
\end{verbatim}

Once we are logged in we have to add our public ssh key to Heroku.

\begin{verbatim}
	$ heroku keys:add
\end{verbatim}

If we would like to delete a key from Heroku we use the following command while providing the corresponding email address. 
\begin{verbatim}
	$ heroku keys:remove example@mail.com 
\end{verbatim}

We can check for any keys linked to the Heroku account in the following way.
\begin{verbatim}
	$ heroku keys 
\end{verbatim}

In order to create a new application in Heroku and link it to our Git repository we use the create command. 
\begin{verbatim}
	$ heroku create
\end{verbatim}

When we want to deploy our application to Heroku we simply need to push it to the heroku branch that has been created in the last step. 
\begin{verbatim}
	$ git push heroku
\end{verbatim}

Using the heroku open command a browser window with our application will be launched.
\begin{verbatim}
	$ heroku open
\end{verbatim}

In order to make sure that heroku is actually launching the correct file we need to edit the package.json file and go to the scripts section. In there we edit the value of the start property to launch the server. 

\begin{lstlisting}
  ...
  "scripts": {
    "test": "echo \"Error: no test specified\" && exit 1",
    "start": "node server.js"  // Heroku will execute this line on startup 
  } 
  ...
\end{lstlisting}


\subsubsection{Making the App Secure}
In order to have a truly secure application we need make sure that no sensitive information like API keys or passwords are stored inside the code. For that matter we create a config.js file that will be added to the .gitignore file to exclude it from being pushed to our repository. Sensitive data should not be pushed to the repository because anybody that has access to the repository will be able to see it.   
 
\begin{lstlisting}
//config.json 
{
  "development": {
    "PORT": 3000, 
    "MONGODB_URI": "mongodb://localhost:27017/WeekMe",
    "JWT_SECRET": "aüaspldapü201122309dxkasdkasldk10"
  }
} 
\end{lstlisting}


One way to remove sensitive information from our code is to use environment variables. 
In the code snipped below we set the environment variable depending on whether we are in our development (localhost) or production (Heroku) environment. 

\begin{lstlisting}
config.js
var env = process.env.NODE_ENV || "development";  
console.log("env ***** ", env);
if(env === "development" || env === "test"){
  var config = require("./config.json"); //When using require on a json file it will automatically be converted to a javascript object.
  var envConfig = config[env]; //get env property of config
  Object.keys(envConfig).forEach((key) => {
    process.env[key] = envConfig[key];
  });
}
\end{lstlisting}
 
Once we have assigned all keys to the process.env variable we may access them from anywhere inside our code. 

\begin{lstlisting}
	var token = jwt.sign({_id: user._id.toHexString(), access}, process.env.JWT_SECRET);
\end{lstlisting}

Last but not least we need to make sure that Heroku is also provided with an environment variable storing the JWT\_SECRET value so that the application may function properly in production.  

	\begin{figure}[H] 
		\centering 
		\includegraphics[height=1.4cm]{figures/herokuenvironmentvar}    
		\caption[Setting an environment variable in Heroku]{Setting an environment variable in Heroku}     
		\label{fig: Setting an environment variable in Heroku}     
	\end{figure}  


\subsubsection{MongoDB \& Mongoose}      
  

\subsection{Frontend}
\subsubsection{Technologies}
\paragraph{Gridstack vs. Bootstrap}
\paragraph{less}
The stylesheets used are generated from less code. 
Leaner Styles Sheets (less) is a backwards-compatible language extension for CSS. 

\subsubsection{Architecture}
\subsubsection{Calling our API}
 

\section{User Documentation}
\textit{(This section is also part of the readme.md on the weekme-github repository)} \\
\\Using \textbf{WeekMe} you can plan your next week. Not more, not less. No bloated UI nor functions you will probably never use anyway. 
How does this \textbf{awesome application} work? Hopefully it's so intuitive that any explanation is unnecessary, but just in case here is a short introduction for you.

\subsection{Account}
Like in most web apps you can sign up for an account, log in and out as well as change your password and email. And of course there is also a solution in case you forget your password. 
\begin{figure}[H] 
	\centering 
	\includegraphics[height=10cm]{figures/user_docu_accounts_page.PNG}   
	\caption[WeekMe account page]{WeekMe account page on an iPhone XS}       
	\label{fig: Setting an environment variable in Heroku}     
\end{figure}  

\subsection{Working with tasks}
Since the app is all about planing your next weeks task, let's show you how you can work with these tasks on WeekMe. 
\subsubsection{Creating tasks}
To create a new task you select the plus icon where you want to add the task. This will bring up a popup window where you can enter a short description (up to 80 characters). 
If you fancy selecting a colour for your task to help you categorise, prioritise or just because you want more colour in your life - no false modesty, go on. Do it. 
In case you changed your mind or just clicked the wrong add button you can also change the day the task will be added to. 

\begin{figure}
    \centering
    \subfloat[Popup to enter description in]{{\includegraphics[height=10cm]{figures/user_docu_createtask.PNG} }}
    \qquad
    \subfloat[Change day later on]{{\includegraphics[height=10cm]{figures/user_docu_createtask_day.PNG} }}
    \caption{Creation process on mobile}
    \label{fig:example}
\end{figure}
 
\subsubsection{Editing tasks}
In case you misspelled something, you want to rephrase something or you just have too much colour on your screen and want to make changes to a task you previously created, just select the task by clicking/tapping on it and select the handy edit button in the upper left corner. 
\subsubsection{Moving tasks}
If you want to bring tasks within one day into order, you can move them around! Just select a task and select the task you want to swap positions with afterwards. 
This also works if you want to move a task to a different day. 
\subsubsection{Checking off and deleting tasks}
You did it. You successfully planned something, stuck to the plan and finished a task. What do you do now? You check the task off by selecting it and clicking/tapping the checkmark icon in the top right corner. It's really that easy! \\
Don't like a task you created anymore? Just check it off as done - no one will know!

\begin{figure}[H] 
	\centering 
	\includegraphics[width=14cm]{figures/user_docu_selected_task_desktop.png}   
	\caption[WeekMe account page]{When selecting a task two buttons will appear}       
	\label{fig: Selected task on desktop}     
\end{figure}  

\subsubsection{The stack}
At the very bottom of your week there is an eighth entry labeled "stack" - What's that all about you might ask - well, this is where you can put tasks that you know you want to do this week but just don't exactly know on what day yet.  You can interact with it just as with every other day as explained above.  
In case you did not manage to check off everything on any given day those tasks will automatically be moved to the stack for you.

\subsubsection{Task creation workflow}
Step by step description on how to create a task. 

	\begin{figure}[H] 
		\centering 
		\includegraphics[height=14cm]{figures/task_creation_workflow}    
		\caption[Task creation workflow]{Setting an environment variable in Heroku}     
		\label{fig: Task creation workflow}     
	\end{figure}  
	
\begin{enumerate}
	\item Click on one of the seven pluses (+) to start the process to create a new task in the corresponding day. 
	\item A popup window will open where you can enter a task-description.
	\item Now you can finish the process by clicking the done button (the task will be added to the day container with the default color (light-yellow). 
	\item Optionally you can click the colored button on the left to open the color-picker.
	\item Choose a color for your task by clicking it or press back. 
	\item Step one will be shown again with an updated color in the color-picker-button.
	\item Now you can modify your description, finish the creation-process or change the day the task will be added to by clicking “another day”.
	\item In this case just choose the day you wish out of a dropdown-menu.
	\item Press “Done” to finish or “Back” if you want to reach step one again to modify your task.
	\item If you want to edit an existing task just click on the task and afterwards on the small pen symbol.
	\item If you click on the check symbol of an task you will finish and delete the task.
\end{enumerate}


% List of Figures, list of tables 
\listoffigures
\addcontentsline{toc}{section}{\numberline{}List of Figures}   
 
\listoftables
\addcontentsline{toc}{section}{\numberline{}List of Tables} 
\cleardoublepage 

\bibliographystyle{IEEEtran}
\bibliography{references/icwref}  
\addcontentsline{toc}{section}{\numberline{}References}       


\end{document}














